\subsection{Условное математическое ожидание в общем случае: определение и свойства. Формула для вычисления условного математического ожидания при известной плотности совместного распределения, условная плотность. Аналог фомрулы Байеса.}
\subsubsection{Условное математическое ожидание в общем случае: определение и свойства.}
Случайная величина вида $f(Y)$ называется \textbf{условным математическим ожиданием} случайной величины X(обладающей математическм ожиданием) относительно случайной велечины Y и обозначается как $E[X|Y]$, если:\\
$$E(Xg(Y)) = E(E(X|Y)g(Y))$$
для всех ограниченных случайных величин $g(Y)$. Любые две случайные величины, удолетворяющие этому определению почти наверное совпадают.\\
Функцию f(y) обозначают как $E(X|Y = y)$ и трактуют как условное математическое ожидание X при условии Y = y. Надо иметь в виду, что именно f(Y) определено однозначно, но не f. Однако различные функции f совпадают почти наверное относительно распределения $m_y$. Если Y имеет положительную непрерывную плотность, то различные функции f совпадают почти всюду. В дальнейшем, если мы пишем $E(X|Y = y)$, то мы имеем в виду $E(X|Y) = f(Y)$.\\
\\
\textbf{Предложение 1.} Сформулированные ранее свойства (i) - (v) условных математических ожиданий для дискретных величин остаются верными и в общем случае.\\
\textit{Доказательство.} Линейность ясна из определения линейности и линейности математического ожидания.\\
Для доказательства монотонности достаточно в силу линейности показать, что изи $X \geq 0$ следует $E(X|Y)\geq 0$ почти наверное. Для этого в определении положим $g(Y) = 1 - sign(E(X|Y)) \geq 0$. Тогда $E(X|Y) - |E(X|Y)| \leq 0$, но:\\
$$E[E(X|Y) - |E(X|Y)|] = E[X(1 - sign(E(X|Y)))] \geq 0$$
Значит $E(X|Y) - |E(X|Y)| = 0$.\\
Равенство $E(E(X|Y)) = EX$ является частным случаем определения($g(Y) = 1$).\\
Если $X$ и $Y$ независимы, то $E(Xg(Y)) = [EX]\cdot [Eg(Y)] = E([EX]\cdot [Eg(Y)])$.\\
Если $Z = h(y)$(с ограниченной h), то подстановкой в определение проверяется, что $ZE(X|Y)$ является условным математическим ожиданием ZX относительно Y. Случай для общей функции h получается с помощью предельного перехода.

\subsubsection{Формула для вычисления условного математического ожидания при известной плотности совместного распределения, условная плотность.}
\textbf{Предложение 2.} Предположим, что распределение $(X, Y)$ задано совместной плотностью $\rho(x,y)_{X, Y}$. Тогда\\
$$E[Ф(X, Y)|Y = y] = \int_{-\infty}^{\infty}Ф(x, y)\frac{\rho(x,y)_{X, Y}}{\rho(y)_{Y}}dx$$
\textit{Доказательство.} Имеет место цепочка неравенств:\\
$E[Ф(X, Y)g(Y)] = \int_{-\infty}^{\infty} \int_{-\infty}^{\infty}Ф(x, y)g(y)\rho(x,y)_{X, Y}dxdy =$\\
$= \int_{-\infty}^{\infty} g(y) (\int_{-\infty}^{\infty}Ф(x, y)\frac{\rho(x,y)_{X, Y}}{\rho_Y(y)}dx)\rho_Y(y)dy$\\
Функцию $\rho_{X|Y}(x|y) = \frac{\rho_{X, Y}(x, y)}{\rho_Y(y)}$ называют условной плотностью X относительно Y(условимся, что она равно 0 в точках y, в которых плотность $\rho_y(y) = 0$). Таким образом верны равенства:\\
$$E(X|Y = y) = \int_{-\infty}^{\infty} x\rho_{X|Y}(x|y)dx, \rho_{X,Y}(x,y) = \rho_{X|Y}(x|y)\rho_Y(y)$$
Последнее из которых является знакомым нам аналогом $P(A\cap B) = P(A|B)P(B)$.

\subsubsection{Аналог фомрулы Байеса.}
Пусть X и Y - такие случайные величины, что существует измеримая функция $\rho(x|y)$, для которой выполнено:\\
$$P(X \in B| Y = y) = \int_B \rho(x|y)dx$$
В этом случае
$$E(h(X)| Y = y) = \int_R h(x)\rho(x|y)dx$$
Заметим, что для произвольной ораниченной функции h
$$Eh(X) = E(E(h(X)|Y))=\int_Rh(x)E\rho(x|Y) dx$$
Тем самым $\rho_X(x) = )E\rho(x|Y)$. Для произвольных ограниченных функция $f, g$ выполнено:\\
$$E[f(X)E(g(Y)|X))]=E[f(X)g(Y)]=Eg(Y)E(f(X)|Y)$$
Левая часть тождества равна
$$\int_R f(x)E[g(y)|X = x)\rho_X(x)dx$$
А правая
$$\int_R f(x)E[g(y)|\rho_X(x))dx$$
В силу произвольности f получаем следующую формулу Байеса:\\
$$E(g(Y)|X = x) = \frac{E[g(Y)\rho(x|Y)}{E\rho(x|Y)}dx$$
Теперь пусть Y принимает значения 0 и 1 с вероятность p и q соотвественно. Тогда:\\
$P(Y = 0|X = x) = \frac{p\rho(x|0)}{p\rho(x|0) + q\rho(x|1)}, P(Y = 1|X = x) = \frac{q\rho(x|1)}{p\rho(x|0) + q\rho(x|1)}$
