% Здесь НЕ НУЖНО делать begin document, включать какие-то пакеты..
% Все уже подрубается в головном файле
% Хедер обыкновенный хсе-теха, все его команды будут здесь работать
% Пожалуйста, проверяйте корректность теха перед пушем

% Здесь формулировка билета
\subsection{Что такое полуинтервал в $\RR^m$? Как определяется простое множество?}
    
    \begin{definition}
       Полуинтервалом в $\RR^m$ называется декартово произведение $m$ полуинтервалов из $\RR$:
       \[ [a^1; b^1) \times [a^2; b^2) \times \dots \times [a^m; b^m) \]
       (при этом $[a; b) \subseteq \RR := \{ x \in \RR \,|\, a \le x < b \}$)
    \end{definition}
    
    \begin{definition}
       Простым множеством называется объединение конечного числа полуинтервалов:
       \[ E = \bigcup\limits_{i=1}^{n} E_i = \bigcup\limits_{i=1}^{n} [a_i; b_i), \]
       где $a_i, b_i$ --- точки в $m$-мерном пространстве.\\
    \end{definition}
