\subsection{Аналитическая функция. Аналитичность голоморфной функции. Неравенство Коши для коэффициентов ряда. Радиус сходимости ряда как максимальный радиус круга, в котором функция голоморфна. Теорема Лиувилля.}

\subsubsection{Аналитическая функция.}

\begin{definition*}
    Функция называется аналитической в точке $z_0$, если $\exists \{c_n\} \in \CC$:

    \[
        f(z) = \sum_{k=0}^{\infty} c_k \cdot (z - z_0)^k \text{, } |z - z_0| < \delta
    \]
\end{definition*}

\subsubsection{Аналитичность голоморфной функции.}

\begin{theorem*}
    Если функция $f$ голоморфна в окрестности $z_0$, то она аналитична в $z_0$.
\end{theorem*}
\begin{proof}
    Пусть $|z - z_0| < \varepsilon < \delta$, $L = \{\zeta: |\zeta - z_0| = \varepsilon\}$

    Тогда по формуле Коши:
    $
    f(z) = \dfrac{1}{2\pi i} \oint_{L} \dfrac{f(\zeta)}{\zeta - z} d\zeta = \dfrac{1}{2\pi i}\oint_{L} \dfrac{f(\zeta)}{\zeta - z_0} \cdot \dfrac{1}{1 - \frac{z - z_0}{\zeta - z_0}} d\zeta = \\ = \dfrac{1}{2\pi i} \oint_{L} \dfrac{f(\zeta)}{\zeta - z_0} \left( 1 + \dfrac{z - z_0}{\zeta - z_0} + \dots + \left(\dfrac{z - z_0}{\zeta - z_0}\right)^n + \dfrac{\left(\dfrac{z - z_0}{\zeta - z_0}\right)^{n + 1}}{1 - \dfrac{z - z_0}{\zeta - z_0}} \right) d\zeta = \\
    = \sum_{k=0}^{n} \left( \dfrac{1}{2\pi i} \oint_{L} \dfrac{f(\zeta)}{(\zeta - z_0)^{k + 1}} d\zeta \right) \cdot (z - z_0)^{k} + \underbrace{\dfrac{1}{2\pi i} \oint_{L} f(\zeta) \cdot \dfrac{\left(\dfrac{z - z_0}{\zeta - z_0}\right)^{n + 1}}{\zeta - z} d\zeta}_{r_n(z, z_0)}
    $

    Пусть $M = \sup_{|\zeta - z_0| = \varepsilon} |f(\zeta)|$, а также заметим, что $|\zeta - z| \geq |\zeta - z_0| - |z - z_0| = \varepsilon(1 - \alpha)$, тогда:
    
    $|r_n(z, z_0)| \leq \dfrac{1}{2\pi}\oint_{L}|f(\zeta)| \cdot \dfrac{\left|\dfrac{z - z_0}{\zeta - z_0}\right|^{n + 1}}{|\zeta - z|} dl \leq \dfrac{M \cdot \alpha^{n + 1}}{2\pi \cdot \varepsilon(1 - \alpha)} \cdot \underbrace{2\pi \varepsilon}_{\text{длина кривой}} \leq \dfrac{M \cdot \alpha^{n + 1}}{\varepsilon(1 - \alpha)} \to 0$ при $n \to \infty$

    Значит, $f(z) = \sum_{k=0}^{\infty} c_k(z - z_0)^{k}$, $c_k = \dfrac{1}{2\pi i} \oint_{L} \dfrac{f(\zeta)}{(\zeta - z_0)^{k + 1}} d\zeta$ при $\forall z: |z - z_0| < \delta$.

    Так как $\dfrac{f(\zeta)}{(\zeta - z_0)^{k + 1}}$ голоморфна в кольце $\varepsilon_1 \leq |z - z_0| \leq \varepsilon_2$, то $\oint_{|z-z_0| = \varepsilon_2} \dfrac{f(\zeta)}{(\zeta - z_0)^{k + 1}}d\zeta - \oint_{|z - z_0| = \varepsilon_1} \dfrac{f(\zeta)}{(\zeta - z)^{k + 1}} d\zeta = 0 \implies c_k$ не зависит от $\varepsilon$.
\end{proof}

\subsubsection{Неравенство Коши для коэффициентов ряда.}

\begin{corollary}
    (Неравенство Коши)

    $|c_k| \leq \oint_{L} \dfrac{|f(\zeta)|}{|\zeta - z_0|^{k + 1}} dl \leq \dfrac{M}{2\pi\varepsilon^{k+1}} \cdot 2\pi\varepsilon = \dfrac{M}{\varepsilon^{k}}$, подставим $M$:

    $|c_k| \leq \dfrac{1}{\varepsilon^{k}} \cdot \sup_{|z - z_0| = \varepsilon} |f(z)|$  $\forall \varepsilon < \delta$
\end{corollary}

\subsubsection{Радиус сходимости ряда как максимальный радиус круга, в котором функция голоморфна.}

\begin{theorem*}
    Пусть $f(z)$ голоморфна в $|z - z_0| < r$, но не является голоморфной в круге большего радиуса, $f(z) = \sum_{k=0}^{\infty} c_k(z - z_0)^k$, $|z - z_0| < R = \dfrac{1}{\overline{\lim_{k \to \infty}} \sqrt[k]{|c_k|}}$. Тогда $R = r$.
\end{theorem*}

\begin{proof}
    Пусть $M = \sup_{|z - z_0| = \varepsilon} |f(z)|$, $|c_k| \leq \dfrac{M}{\varepsilon^{k}}$  $\forall \varepsilon < r$

    $\overline{\lim}\sqrt[k]{|c_k|} \leq \dfrac{1}{\varepsilon} \implies R \geq \varepsilon$  $\forall \varepsilon < r \implies R \geq$.

    Но если $R > r$, то ряд сходится в $z$: $|z - z_0| > r$, что противоречит условию.

    Значит, $R = r$.
\end{proof}

\subsubsection{Теорема Лиувилля.}

\begin{theorem*}
    (Лиувилля)

    Если функция $f(z)$ голоморфна и ограничена на $\CC$, то она -- константа.
\end{theorem*}

\begin{proof}
    Пусть $M = \sup_{z \in \CC} |f(z)|$, $f(z) = \sum_{k=0}^{\infty} c_k z^{k}$

    Так как $|c_k| \leq \dfrac{M}{\varepsilon^{k}}$  $\forall \varepsilon$, то при $\varepsilon \to \infty$ получаем, что $c_1 = c_2 = \dots = 0 \implies f(z) = c_0$ 
\end{proof}
